% -*- ispell-dictionary: "french" -*-

\begin{frame}[fragile]{Un peu d'histoire}
  \begin{description}
  \item[1972 :] première version de \texttt{C}
  \item[1983 :] première version de \texttt{C++} (alors \texttt{C with classes})
  \item[1989 :] premier standard pour \texttt{C} (\texttt{ANSI C})
  \item[1998 :] premier standard pour \texttt{C++}
  \item[2003 :] norme \texttt{C++03}
  \item[2011 :] norme \texttt{C++11} $\leftarrow$ début du \texttt{C++} moderne
  \item[2014 :] norme \texttt{C++14}
  \item[2017 :] norme \texttt{C++17}
  \end{description}
  \begin{lstlisting}[language=bash]
    --std=c++11
  \end{lstlisting}
\end{frame}

\begin{frame}[fragile]{De nouveaux tableaux}
  \begin{lstlisting}
std::array<int,5> a;
a[1] = 2;
size_t a_size = a.size();

a[6] = 4; // runtime error
  \end{lstlisting}
\end{frame}

\begin{frame}[fragile]{De nouveaux tableaux}
  \begin{lstlisting}
std::vector<int> a;
a.push_back(10);
a[0];

std::vector<int> b (5);
b.pop_back();
size_t b_size = b.size(); // now 4
  \end{lstlisting}
\end{frame}

\begin{frame}[fragile]{Une nouvelle boucle \texttt{for}}
  \begin{lstlisting}
std::array<int,5> a;
// code
for (int &it : a) {
  // code
}
  \end{lstlisting}
  \texttt{range for} pour tous les itérables
  \begin{itemize}
  \item \texttt{begin} et \texttt{end} sur l'objet
  \item \texttt{operator!=}, \texttt{operator++} et \texttt{operator*} sur l'itérateur
  \end{itemize}
\end{frame}

\begin{frame}[fragile]{Le type \texttt{auto}}
  \texttt{auto} : inférence de type
  
  \begin{lstlisting}[escapechar=ù]
for (auto &iterator : ù\textit{iterable}ù) {
  // code
}
  \end{lstlisting}
\end{frame}

\begin{frame}[fragile]{Parlons performance}
  Que se passe-t-il si j'exécute le code ci-dessous ?
  
  \begin{lstlisting}
std::string f () {
  // code
}

int main () {
  std::string x, y;
  // put some text in x and y
  std::string a(x);
  std::string b (x+y);
  std::string c (f());
  return 0;
}
  \end{lstlisting}
\end{frame}

\begin{frame}[fragile]{Parlons performance}
  Que se passe-t-il si j'exécute le code ci-dessous ?
  
  \begin{lstlisting}
std::vector<std::string> f () {
  // return a large vector of long strings
}

int main () {
  std::vector<std::string> a (f());
  return 0;
}
  \end{lstlisting}
\end{frame}

\begin{frame}{Parlons performance}
  Pourquoi faire une copie ?

  $\implies$ Indiquer au compilateur que l'objet affecté est détruit immédiatement après !
\end{frame}

\begin{frame}[fragile]{\textit{Move semantics}}
  \begin{itemize}
  \item Possible depuis \texttt{C++11}
  \item Utilisation automatique
  \item Utilisation manuelle avec \texttt{std::move}
  \end{itemize}
  
  \begin{lstlisting}[escapechar=ù]
ù\textit{new\_var}ù = std::move(ù\textit{old\_var}ù)
  \end{lstlisting}
\end{frame}

\begin{frame}[fragile]{\textit{Move semantics}}
  \begin{lstlisting}
class A {
  // stuff
public:
  ~A ();
  A (const A &);
  A (A &&); // move constructor
  A& operator= (const A &);
  A& operator= (A &&); // move assignment operator
  // more stuff
};
  \end{lstlisting}
\end{frame}

\begin{frame}{\textit{Move semantics}}
  Autant que possible, utiliser les méthodes par défaut !
\end{frame}

\begin{frame}[fragile]{\textit{Move semantics} en pratique}
  \begin{lstlisting}
class blob {
private:
  size_t m_size;
  char *m_data;
public:
  blob (): m_size(0), m_data(nullptr) {}
  ~blob () { delete[] m_data; }
};
  \end{lstlisting}
\end{frame}

\begin{frame}[fragile]{\textit{Move semantics} en pratique}
  \begin{lstlisting}
blob (const blob& other): m_size(other.m_size) {
  // copy constructor
  if (m_size) {
    m_data = new char[m_size];
    memcpy(m_data, other.m_data, m_size);
  } else {
    m_data = nullptr;
  }
}
  \end{lstlisting}
\end{frame}

\begin{frame}[fragile]{\textit{Move semantics} en pratique}
  Avec \textit{copy-and-swap} :
  
  \begin{lstlisting}
friend void swap (blob& first, blob& second) {
  using std::swap;
  //
  swap(first.m_size, second.m_size);
  swap(first.m_data, second.m_data);
}

blob (blob&& other): blob() {
  // move constructor
  swap(*this, other);
}
  \end{lstlisting}
\end{frame}

\begin{frame}[fragile]{Retour vers les ressources}
  \begin{itemize}
  \item Ressource (encapsulée dans un objet) \og{}unique\fg{}
  \item Interdire la copie ?
  \end{itemize}

  \begin{lstlisting}
class A {
  // stuff
public:
  ~A ();
  A (const A &) = delete;
  A (A &&) = delete;
  A& operator= (const A) = delete;
};
  \end{lstlisting}

  \textit{delete} utilisable aussi avec l'héritage
\end{frame}

\begin{frame}[fragile]{Retour vers les ressources}
  Impossible d'utiliser une \textit{factory} ?
  
  \begin{lstlisting}
class A {
  // stuff
public:
  ~A ();
  A (const A &) = delete;
  A (A &&);
  A& operator= (const A);
};
  \end{lstlisting}
\end{frame}

\begin{frame}[fragile]{Retour vers les ressources}
  \begin{lstlisting}
int main () {
  A a = factory_for_A();
  A b, c;
  b = a; // Forbidden
  c = std::move(a); // a may end up unusable
  return 0;
}
  \end{lstlisting}
\end{frame}

\begin{frame}[fragile]{Les \textit{lambda}s}
  \begin{lstlisting}
void f (int i) {
  std::cout << i << ' ';
}

int main () {
  std::vector<int> vec;
  vec.push_back(10);
  // ...
  for_each(vec.begin(), vec.end(), f);
  std::cout << '\n';
  return 0;
}
  \end{lstlisting}
\end{frame}

\begin{frame}[fragile]{Un exemple de \textit{lambda}}
  \begin{lstlisting}
int main () {
  std::vector<int> vec;
  vec.push_back(10);
  // ...
  for_each(vec.begin(), vec.end(), [](int i) {
      std::cout << i << ' ';
    } );
  std::cout << '\n';
  return 0;
}
  \end{lstlisting}
\end{frame}

\begin{frame}[fragile]{{Les \textit{lambda}s}}
  \begin{lstlisting}
[](double x) -> double {
  if (x>0) {
    return 0.1*x;
  } else {
    return 0;
  }
}
  \end{lstlisting}
\end{frame}

\begin{frame}[fragile]{{Les \textit{lambda}s}}
  Capture de variables :
  \begin{itemize}
  \item \texttt{[}\textit{var}\texttt{]} pour capturer par valeur
  \item \texttt{[\&}\textit{var}\texttt{]} pour capturer par référence
  \item \texttt{[]} pour capturer toutes las variables par valeur
  \item \texttt{[\&]} pour capturer toutes las variables par référence
  \item \texttt{[=, \&}\textit{var}\texttt{]}
  \item \texttt{[\&, }\textit{var}\texttt{]}
  \end{itemize}

  Un dernier mot-clé : \texttt{mutable}
\end{frame}
