% -*- ispell-dictionary: "french" -*-

\begin{frame}[fragile]{Les \texttt{string}s}
  \texttt{string} : chaîne de caractères améliorée
  \begin{lstlisting}
#include <string>
    
int main () {
  std::string str ("some");
  std::string str2 ("text");
  str.append(str2);
  const char *cstr (str.c_str());
  return 0;
}
    \end{lstlisting}
\end{frame}

\begin{frame}[fragile]{Structures et objets}
  \begin{lstlisting}
struct atom {
  int atomic; // Atomic number
  int mass; // Mass number
};
  \end{lstlisting}

  Membres d'une fonction :
  \begin{itemize}
  \item variables (attributs),
  \item fonctions (méthodes)
  \end{itemize}
\end{frame}

\begin{frame}[fragile]{Structures et objets}
  \begin{lstlisting}
atom carbon12;
carbon12.atomic = 6;
carbon12.mass = 12;

atom *oxygen16 = new atom;
oxygen16->atomic = 8;
oxygen16->mass = 16;
  \end{lstlisting}
\end{frame}

\begin{frame}[fragile]{Structures et objets}
  \begin{lstlisting}
class Atom {
private:
  int m_atomic;
  int m_mass;
public:
  Atom(int a, int m) {
    m_atomic = a; m_mass = m;
  }
};

atom carbon12 (6,12);
carbon12.m_mass = 13; // Forbidden!
  \end{lstlisting}
\end{frame}

\begin{frame}{Sructures et objets}
  \textcolor{red}{Les attributs d'une classe sont toujours \textbf{privés}.}
\end{frame}

\begin{frame}[fragile]{Un objet minimaliste}
  \begin{lstlisting}
class A {
private:
  // attributes
  // private methods
public:
  ~A (...) {...} // destructor
  A (const A &a) {...} // copy constructor
  A& operator= (const A &a) {...} // copy assignment operator
  // public methods
};
  \end{lstlisting}
\end{frame}

\begin{frame}[fragile]{Un exemple}
  \begin{lstlisting}
class GaussInt {
private:
  int m_real;
  int m_imag;
public:
  GaussInt (int a=0, int b=0): m_real(a), m_imag(b) {}
  ~GaussInt () {}
  GaussInt (GaussInt &a): m_real(a.m_real), m_imag(a.m_imag) {}
  double norm () { return std::sqrt(m_real*m_real + m_imag*m_imag); }
};
  \end{lstlisting}
\end{frame}

\begin{frame}[fragile]{Un exemple}
  \begin{lstlisting}
int main () {
  GaussInt a;
  GaussInt b (4,3);
  GaussInt c (b);
  GaussInt d = a;
  std::cout << "Norm of c: " << c.norm() << std::endl;
  std::cout << "Norm of d: " << d.norm() << std::endl;
  return 0;
}
  \end{lstlisting}
\end{frame}

\begin{frame}[fragile]{Méthodes par défaut}
  \begin{itemize}
  \item Définition implicite du constructeur de copie, de l'opérateur d'affectation et du destructeur
  \item Copie membre à membre
  \end{itemize}

  \begin{lstlisting}
class GaussInt {
  // ...
public:
  GaussInt (int a=0, int b=0): m_real(a), m_imag(b) {}
  double norm () { return std::sqrt(m_real*m_real + m_imag*m_imag); }
};
  \end{lstlisting}
  
  Implémentation seulement si nécessaire !
\end{frame}

\begin{frame}{Les possibilités}
  \begin{itemize}
  \item Surcharger les opérateurs
  \item Accès aux membres : \texttt{private}, \texttt{protected} et \texttt{public}
  \item Héritage (multiple)
  \item Méthodes \texttt{const}, attributs \texttt{static} et \texttt{mutable}
  \item Fonctions et classes amies (\texttt{friend})
  \end{itemize}
\end{frame}

\begin{frame}{Séparation en fichiers}
  \begin{itemize}
  \item Fichiers \textit{header} (extension \texttt{h} ou \texttt{hpp}) : déclarations
  \item Fichiers source (extension \texttt{cpp}) : définitions
  \end{itemize}

  On inclut des fichiers \textit{header} dans les fichiers source ou dans d'autres fichiers \textit{header}.
\end{frame}

\begin{frame}[fragile]{Séparation en fichiers : \texttt{GaussInt.h}}
  \begin{lstlisting}
#ifndef GaussInt_included
#define GaussInt_included
class GaussInt {
private:
  int m_real;
  int m_imag;
public:
  GaussInt (int a=0, int b=0);
  ~GaussInt ();
  GaussInt (const GaussInt &);
  GaussInt& operator= (const GaussInt &);
  double norm ();
};
#endif
  \end{lstlisting}
\end{frame}

\begin{frame}[fragile]{Séparation en fichiers : \texttt{GaussInt.cpp}}
  \begin{lstlisting}
#include "GaussInt.h"
#include <cmath>
GaussInt::GaussInt (int a=0, int b=0): m_real(a), m_imag(b) {}
GaussInt::~GaussInt () {}
GaussInt::GaussInt (const GaussInt &a): m_real(a.m_real), m_imag(a.m_imag) {}
  \end{lstlisting}
\end{frame}

\begin{frame}[fragile]{Séparation en fichiers : \texttt{GaussInt.cpp}}
  \begin{lstlisting}
GaussInt& GaussInt::operator= (const GaussInt &a) {
  m_real = a.m_real; m_imag = a.m_imag;
  return *this;
}
double GaussInt::norm () {
  return std::sqrt(m_real*m_real + m_imag*m_imag);
}
  \end{lstlisting}
\end{frame}

\begin{frame}[fragile]{Séparation en fichiers}
  \begin{lstlisting}
#include <iostream>
#include "GaussInt.h"
int main () {
  GaussInt a;
  GaussInt b (4,3); GaussInt c (b);
  GaussInt d; d = c;
  std::cout << "Norm of a: " << a.norm() << std::endl;
  std::cout << "Norm of c: " << c.norm() << std::endl;
  std::cout << "Norm of d: " << d.norm() << std::endl;
  return 0;
}
  \end{lstlisting}
\end{frame}

\begin{frame}[fragile]{Compilation}
  \begin{lstlisting}[language=bash]
g++ -Wall -c GaussInt.cpp -o GaussInt.o
g++ -Wall -c main.cpp -o main.o
g++ -Wall main.o GaussInt.o -o gauss.out
  \end{lstlisting}

  On peut exécuter :
  \begin{lstlisting}
./gauss.out
Norm of a: 0
Norm of c: 5
Norm of d: 5
  \end{lstlisting}
\end{frame}

\begin{frame}[fragile]{Les \textit{template}s}
  \begin{lstlisting}
template <typename T> bool isPositive (T a) {
  if (a>=0) {
    return true;
  } else {
    return false;
  }
}
  \end{lstlisting}

  Mot-clé : \texttt{typename} ou \texttt{class}
\end{frame}

\begin{frame}[fragile]{Les \textit{template}s}
  \begin{lstlisting}
template <typename T> class complex {
private:
  T m_real;
  T m_imag;
public:
  complex (const T &a = T(), const T &b = T());
  ~complex();
  complex (const complex<T> &);
  complex& operator= (const complex<T> &);
  double norm ();
};
  \end{lstlisting}
\end{frame}

\begin{frame}[fragile]{Les \textit{template}s}
  \begin{lstlisting}
int n = 4;
double x = 2.5;
isPositive(n);
isPositive(x);
isPositive<double>(3.0);
  \end{lstlisting}

  \begin{lstlisting}
complex<double> z (3, 5.5);
  \end{lstlisting}
\end{frame}

\begin{frame}{Les \textit{template}s}
  \begin{itemize}
  \item  Des patrons de classe ou de fonction pour générer automatiquement du code
  \item Plusieurs paramètres, avec des valeurs par défaut
  \item Spécialisation de \textit{template}s
  \end{itemize}

  Déclaration et définition \textbf{dans le même fichier} $\implies$ dans un \textit{header}
\end{frame}

\begin{frame}{Une application de la POO : RAII}
  \begin{itemize}
  \item Fuites : mémoire, fichiers, connections\dots{}
  \item \texttt{gdb}, \texttt{valgrind}\dots{}
  \item En cas d'interruption du programme (par exemple \texttt{exception})?
  \end{itemize}
\end{frame}

\begin{frame}{Une application de la POO : RAII}
  \begin{itemize}
  \item RAII : Ressource Acquisition Is Initialisation
  \item Encapsulation des ressources dans des objets
  \item[$\rightarrow$] Initialisation dans le constructeur, nettoyage dans le destructeur
  \item[$\rightarrow$] Appel du destructeur garantit par le langage
  \item[$\rightarrow$] Exception renvoyée par un destructeur : \textit{undefined behaviour}
  \end{itemize}
\end{frame}
